%%%%%%%%%%%%%%%%%%%%%%%%%%%%%%%%%%%%%%%%%%%%%%%%%%%%%%%%%%%%%%%%%%%%%%%%%%%%%%%%
% Template for ASPLOS papers.
%
% History:
% 
% ASPLOS originally used jpaper.cls for submission but required acmart.cls for the
% final camera-ready version. To avoid a change in format, starting ASPLOS 2024 Fall 
% cycle, both the submission and the camera-ready versions started using acmart.cls.
%
%%%%%%%%%%%%%%%%%%%%%%%%%%%%%%%%%%%%%%%%%%%%%%%%%%%%%%%%%%%%%%%%%%%%%%%%%%%%%%%%%%

% use the base acmart.cls version 1.92
% use the sigplan proceeding template with the default 10 pt fonts
% nonacm option removes ACM related text in the submission. 



\documentclass[nonacm,sigplan]{acmart}

\usepackage{amsmath,amsfonts}
\usepackage{physics}
\usepackage{algorithmic}
\usepackage{graphicx}
\usepackage{textcomp}
\usepackage{xcolor}
\usepackage{svg}
\usepackage{tikz}
\usetikzlibrary{quantikz}
\usepackage{caption}
\usepackage{subcaption}

% enables page numbers
\settopmatter{printfolios=true}

\begin{document}

\title{Realtime Compilation for Continuous Angle Quantum Error Correction Architectures}

%\author{...} % removed for anonymity

\begin{abstract}
Quantum error correction (QEC) is necessary to run large scale quantum programs. Regardless of error correcting code, hardware platform, or systems architecture, QEC systems are limited by the types of gates which they can perform efficiently. In order to make the base code's gate set universal, they typically rely on the production of a single type of resource state, commonly T, in a different code which is then distilled and injected into the base code. This process is neither space nor time efficient and can account for a large portion of the total execution time and physical qubit cost of any program. In order to circumvent this problem, alternatives have been proposed, such as the production of continuous angle rotation states \cite{akahoshi2023partially, ding2018magic}. These proposals are powerful because they not only enable localized resource generation but also can potentially reduce total space requirements. 

However, the production of these states is non-deterministic and can require many repetitions in order to obtain the desired resource. The original proposals suggest architectures which do not actively account for realtime management of its resources to minimize total execution time. Without this, static compilation of programs to these systems will be unnecessarily expensive. In this work, we propose a realtime compilation of programs to these continuous angle systems and a generalized resource sharing architecture which actively minimizes total execution time based on expected production rates. To do so, we repeatedly redistribute resources on-demand which depending on the underlying hardware can cause excessive classical control overhead. We further address this by dynamically selecting the frequency of recompilation. Our compiler and architecture improves over the baseline proposals by an average of $4.5\times$.
\end{abstract}

\maketitle % should come after the abstract
\pagestyle{plain} % should come right after \maketitle

\section{Introduction}
\begin{comment}
1. QEC good
2. QEC for Surface Code Appealing for near-term architectures
3. QEC doesn't have a universl gate set inherently
4. Must augment the gate set of the base code with gates from other codes; e.g. T or CCX. THis is done via distillation or code switching
5. Then, use synthesis procedures to convert Rz to S-H-T. This is expensive 
6. Recently new proposals have appeared: generate Rz gates directly \cite{}
7. how does it work?
8. Architectural Support for them is lacking
9. Direct accounting for the non-deterministic behavior of their production and consumption (and possibly long corrections) missing
\end{comment}
\begin{figure*}
    \centering
    \includesvg[width=\textwidth]{figs/intro.svg}
    \caption{Sample execution patterns of STAR \cite{akahoshi2023partially} (top) compared to our more dynamic and efficient resource management compiler (bottom). In STAR, atomic units of 2x2 grids are allocated containing 1 data qubit and 3 ancilla. Continuous angle rotation states $\ket{m_\theta}$ can be prepared locally and consumed within each STAR unit. However, ancilla use is locally restricted as well which means non-deterministic state production can take longer than necessary without utilizing parallel preparations with neighboring ancilla. Our compiler effectively manages ancilla for both state preparation and communication based on dynamically and in realtime computed expected free times for each resource and allow for multiple parallel preparations to minimize total expected execution time, here 6 preparations are attempted in parallel (bottom) as opposed to 1 (top), this essentially guarantees a single cycle to produce the necessary state. This leads to on average $4.5\times$ improvement in total runtime.}
    \label{fig:intro}
\end{figure*}
Current quantum computers very noisy and only able to reliably execute small programs. In order to combat high error rates, quantum error correction is necessary in order to execute large scale applications. Even still, as we scale, quantum resources will be relatively scarce and it is critical to develop real time management of these resources in order to maximize the value of available hardware. These resources include classical bandwidth for decoding \cite{ravi2023better}, ancilla management for communication and decompositions \cite{ding2020square}, and the creation and use of special resource states called \textit{magic states} \cite{ding2018magic, litinski2019magic}. So far, the study of these resources has been concentrated around surface codes as a promising candidate error correction code for small to intermediate scale quantum hardware \cite{litinski2019game, gidney2021factor} due to its limited hardware connectivity, high threshold, and well studied decoding procedures. While surface codes are not definitively the best choice of error correction code, for example qLDPC codes \cite{bravyi2023high, xu2023constant} are a promising choice for high rate codes, they are fairly easy to work since they have well-defined logical operators via lattice surgery \cite{horsman2012surface} and can serve as as testbed for different architectural designs.

Critical to the success of any quantum error correcting code is support for the creation and consumption of so-called \textit{magic states}. Since no code natively supports a transversal and universal gate set \cite{eastin2009restrictions}, there must be some gates that are performed in \textit{another} code and converted or distilled into special states that can be used to apply that gate in the original code. For example, in the surface code we natively only support the Clifford gates (e.g. CNOT, X, Z, and H) which are not universal. One common way to make this gate set universal is to produce T states and inject them as necessary. It is also typical that we only support a finite gate set and not a continuous one. For example, most physical quantum computers can support $Rz(\theta)$ for any $\theta$, where as the surface code only supports a discrete set. However, by adding in the T gate to the Clifford we do get a universal gate set since we can approximate any $Rz(\theta)$ gate to arbitrary precision \cite{ross2014optimal}. Unfortunately, this type of synthesis results in circuits which are extremely long and require large numbers of bulky \textit{factories} to produce the magic states; both space and time requirements for any circuit become dominated by T state production. 

Recently, several alternative approaches have been proposed \cite{akahoshi2023partially, choi2023fault} which rather than requiring magic state factories to create one fixed type of Rz rotation instead propose methods to create analog rotation states $\ket{m_\theta}$, adopting the syntax from \cite{akahoshi2023partially}. When injecting this state we can perform an arbitrary $Rz(\theta)$ rotation. This is especially powerful because it ideally reduces the total space requirement for producing non-Clifford gates since we can use ancilla local to the injection site to prepare the necessary $\ket{m_\theta}$ and also does not require us to distill hundreds or thousands of T gates per $Rz$ rotation. So far, however, there is limited architectural support for this strategy specifically for the realtime management of the nondeterministic behavior of the $\ket{m_\theta}$ production. In the original proposal from \cite{akahoshi2023partially} their STAR architecture provides a basic structure to demonstrate the technique's efficacy, but 1. limits state production to atomic STAR patches which limits parallel production when other space is unused and 2. does not directly adapt program execution as a function of the highly non-deterministic state production. 

In this work, we provide an improved compilation scheme for continuous angle rotation architectures. In our compilation technique, we consider the ancillas independent of the data qubit, thus allowing for sharing of resources between multiple qubits and gates. This warrants the need to make decisions on the dynamic assignment of ancilla since allocating more ancilla for a particular gate operation reduce s the expected time to prepare $\ket{m_\theta}$ but constraints the execution of neighboring gates, whereas allocating fewer ancilla leads to inefficient utilisation of the resources. To counter this problem, we provide a mechanism to dynamically manage the ancilla that are available for different gate operations, including but not limited to Rz and CNOT gates. This approach allows us to flexibly allocate the ancilla for different gates. We also propose an space-time efficient technique that schedules the long-distance gate operations while minimizing the wait time. We achieve an average improvement of $4.5\times$ over a statically compiled execution, even after accounting for classical overheads during dynamic compilation to recompute the best ancilla use. Our approach can be directly incorporated in any quantum architecture involving non-deterministic execution and/or variable ancilla availability.

The primary contributions of this work are
\begin{enumerate}
    \item An efficient compilation scheme for quantum error correction systems which support native continuous angle resource states. We improve over baseline proposals by an average of $4.5\times$ in total program execution time, both quantum and classical delay costs accounted for.
    \item An improved architecture for local resource state production which reduces total space (ancilla) requirements while simultaneously reducing the total runtime of programs on these systems.
    \item Compilation which directly accounts for the inherent non-deterministic behavior of continuous angle resource state production; we introduce the notion of real-time recompilation depending on the prior success of production and consumption of these states.
    \item Real system measurement latencies restrict the amount and frequency of classical recompilation that can occur without incurring excessive idling on the quantum system; our compilation scheme easily adapts to any hardware platform and dynamically selects the frequency of recompilation the first of its kind to do so. This real-time control consideration is extensible to other QEC systems.
\end{enumerate}
\section{Background}
\begin{figure}
    \centering
    \includesvg[width=0.5\columnwidth]{figs/surface_code.svg}
    \caption{A surface code patch with $d = 3$. Data (black) interact with ancilla (white) to extract syndrome information used to decode errors. The rate at which these syndromes can be measured depends on the system and determines the duration of a single \textit{code cycle} which influences how much time can be used for realtime recompilation.}
    \label{fig:surface_code}
\end{figure}
\subsection{Quantum Error Correction}
% Very basics of QEC
% What are logical qubits / what are logical operations
In \textit{physical} quantum computing, physical qubits are linear superpositions of basis states as $\ket{\psi} = \alpha\ket{0} + \beta\ket{1}$. In contrast, many physical qubits can be used to create a \textit{logical} qubit which encodes information as superpositions of a logical basis $\overline{\ket{\psi}} = \alpha\overline{\ket{0}} + \beta\overline{\ket{1}}$. Quantum systems are inherently prone to noise due to the interaction between the physical object used for the qubit and the surrounding environment (e.g. ions or atoms manipulated via lasers). The frequency of errors in physical systems is nontrivial, often occurring on the order of 1 in every 100 to 1000 operations. This is far from necessary error rates for large-scale computation. By encoding information into logical qubits, error rates can be suppressed arbitrarily corresponding to the code's distance at the cost of additional space. There are a large variety of potential quantum error correcting codes. For example, one popular code is the Surface Code which has a \textit{threshold} of about 1\% \cite{fowler2018low}, the physical error rate below which increasing the code's distance results in exponential reduction in error rates. 

It is important to distinguish the physical error rate of the underlying physical qubits versus the logical error rate of the encoded system. During the maintenance of the logical quantum state which can't be measured exactly without destroying the state, we carefully extract some information from the system, called syndromes, which are passed to a decoder in order to determine what physical errors have happened either due to idling or due to the execution of gates. The processes of gathering syndromes and decoding, while important, can often be abstracted away from the consideration of the total logical system because they are performed repeatedly and without any modification over the course of the entire program.

Logical quantum information can similarly be manipulated like physical information. To manipulate physical quantum systems, gates are performed which modify the coefficients $\alpha$ and $\beta$. For some codes, logical operations are simply a collection of physical gates on its constituent physical qubits. For example, a \textit{transversal} logical X gate might be performed by simply performing physical X gates on every data qubit. Some codes have more exotic logical operators, which although they consist of some collection of physical operations, are not as straightforward as these transversal gates. 

\subsection{Surface Codes}

% Specifics of this code that are relevant to this work
% E.g. dxd patches, X edge, Z edges
% Lattice Surgery Operations - Pauli Product Measurements / long-range CNOTs
In this work, we are primarily focused on Surface Codes a type of topological code with many attractive properties for available or soon-to-be available hardware platforms. Specifically, surface codes require only nearest neighbor connectivity between physical qubits, its parity checks require only 4 two qubit gates, it has a high threshold and it has well-studied decoders \cite{Fowler_2012, higgott2022pymatching}. Surface code architectures are also fairly well studied with the most popular being the rotated surface code with lattice surgery operations. In it's most simple form, the rotated surface code is a $d \times d$ patch of qubits with $O(d^2)$ data qubits and ancilla qubits, where $d$ is the distance of the code. The patch is composed of $X$ checks and $Z$ checks which collect syndromes about $X$ and $Z$ errors, respectively. The boundaries of the square patch are either $X$ or $Z$ edges which determine how the logical qubit can interact with other qubits. Figure~\ref{fig:surface_code} shows a logical qubit surface code for $d=3$.


Surface code architectures are simply a fabric of many $d \times d$ sized tiles where program qubits are mapped onto these tiles. Tiles can be deformed fairly arbitrarily, so long as the distance of the shortest path between any pair of corresponding edges (e.g. X edge to another X edge) is always at least $d$. Deformation, amongst other operations, is one of many available operations in a surface code architecture and can be used to transport information from one region of the device to another in order to interact with other distance logical qubits. In order to do so, the grid of tiles must allocate some number of logical \textit{ancilla} which are virtually empty. In order to interact two logical qubits, we can employ so-called lattice surgery which, using an intermediate ancilla. For a more complete coverage of these operations, we refer to \cite{litinski2019game}.

\begin{figure*}
    \centering
    \begin{subfigure}{0.24\textwidth}
        \includesvg[width=\linewidth]{figs/lattice_cnot_1.svg}
        \caption{Initial state}
    \end{subfigure}
    \begin{subfigure}{0.24\textwidth}
        \includesvg[width=\linewidth]{figs/lattice_cnot_2.svg}
        \caption{Merge step (1 cycle)}
    \end{subfigure}
    \begin{subfigure}{0.24\textwidth}
        \includesvg[width=\linewidth]{figs/lattice_cnot_3.svg}
        \caption{Split + Merge step (1 cycle)}
    \end{subfigure}
    \begin{subfigure}{0.24\textwidth}
        \includesvg[width=\linewidth]{figs/lattice_cnot_4.svg}
        \caption{Split step (0 cycles)}
    \end{subfigure}
    \caption{Execution of a CNOT gate in lattice surgery using exactly 2 steps. Here the CNOT is performed with a single ancilla patch in between. In general, this patch can be arbitrarily long and shaped so long as the correct boundaries are adjacent. We assume ancilla prepartion is done ahead of time.}
    \label{fig:lattice_cnot}
\end{figure*}

We summarize the necessary information for this work:
\begin{itemize}
    \item Program qubits are assigned to a single $d \times d$ tile in the larger fabric
    \item The logical operation CNOT (Figure~\ref{fig:lattice_cnot}) occurs in \textbf{two} steps: a. measure the ZZ operator between the control and the ancilla followed by b. measure the XX operator between the ancilla and the target.
    \item Interactions between logical qubits use a contiguous path of ancilla qubits which must touch every interacting qubit
    \item A special type of multi-qubit interaction, or \textit{Pauli-product measurement}, can in \textbf{one} step by interacting the $P$-edge of each interacting logical qubit to an intermediate ancilla channel contiguous between each, where $P$ is the specific Pauli. For example, a ZZ Pauli product measurement can be done by interacting the Z edge of qubit 1 and the Z edge of qubit 2 to one contiguous block of ancilla. This can be done in 1 step regardless of distance between them, so long as the ancilla channel is contiguous.
\end{itemize}

\subsection{Compilation: Physical vs. Logical}
In this work, we want to clearly distinguish the compilation of programs at the physical level from the compilation of programs at the logical level. Specifically, for the surface code, we consider \textit{physical} compilation to be the process by which the data and ancilla qubits and physical gates of the logical qubit are mapped, routed, and scheduled on the specific hardware \cite{wu2021mapping, viszlai2023architecture, leblond2023tiscc}. For example, if the target hardware is trapped ions the exact execution of specific syndrome measurements is determined by physical compilation and is otherwise unimportant at the higher program level. In this work, we focus on \textit{logical compilation} which refers to the mapping, routing, and scheduling of program qubits onto logical qubits \cite{hua2021autobraid, litinski2019game, litinski2019magic}. In order to be platform agnostic, we consider code cycles during which an entire round of low-level physical operations are performed. Therefore, our basic units are logical qubits and code cycles as opposed to physical devices and execution time in seconds. % We later will consider the practical ramifications of platform selection for dynamic recompilation.

\subsection{Clifford + T, Synthesis}
All quantum error correction codes do \textit{not} support a set of universal set of logical operators. Codes admit a partially complete set of gates; for example, surface codes can perform the Clifford gates easily which includes $\langle H, S, X, Z, CNOT \rangle$, the set generated by these gates. However, Cliffords are not universal and need to be augmented by some additional non-Clifford gate. The two most typical are T or Toffoli (CCX) gates and all gates in the input program must be synthesized into the gate sets Clifford+T or Clifford+CCX. For example, if an input program contains $Rz(\theta)$ rotations they must be decomposed into a finite sequence of $\{H, S, T\}$ as a function of approximation precision. These sequences can often take up the bulk of large-scale programs. For example, precision on the order of $10^{-8}$ or smaller is necessary and each require hundreds if not thousands of gates. Because $T$ is not native, it must be prepared remotely which requires a large amount of space and time to prepare.

\subsection{Resource State Distillation}
In order to make gate sets for quantum error correction codes, additional resource states (for example $\ket{T}$ or T states which can be used to perform logical T gates via teleportation) are prepared remotely in a different code which does admit a transversal implementation, for example the Quantum Reed Muller codes. The preparation of these states, however, can fail and are prepared with some fixed logical error rate which is different from the logical error rate of the base code. However, through a process of distillation more error prone states can be used to generate lower logical error rate versions. Resource states are usually prepared remotely in factories and consumed on demand by the program. The total space-time cost of factory distillation in some estimates can take upwards of 90\% of the total space-time volume of the program's execution. Alternatives to this paradigm could reduce this overhead tremendously.

\section{Related Work}\label{section:relatedwork}
% \subsection{Small Angle Rotations}\label{section:smallanglerotations}
% Yongshan + Japan
\begin{figure*}
    \centering
    \begin{subfigure}{0.24\textwidth}
        \centering
        \includesvg[width=\linewidth]{figs/regular_star.svg}
    \end{subfigure}
    \begin{subfigure}{0.24\textwidth}
        \centering
        \includesvg[width=\linewidth]{figs/compact_star.svg}
    \end{subfigure}
    \begin{subfigure}{0.24\textwidth}
        \centering
        \includesvg[width=\linewidth]{figs/compressed_star.svg}
    \end{subfigure}
    \begin{subfigure}{0.24\textwidth}
        \centering
        \includesvg[width=\linewidth]{figs/mixed_grid.svg}
    \end{subfigure}
    \caption{Different patches proposed in the STAR architecture and the grid on the right shows how we `insert' the compact and compressed patches into a 2D grid of regular STAR patches. The different colours show the regular STAR patch with the darker colour indicating the data qubit and the lighter colours indicating ancilla qubits. The patches with red outline are the compact and compressed patches.}
    \label{fig:star}
\end{figure*}
Recent works \cite{akahoshi2023partially, ding2018magic} have proposed fault-tolerant architectures to prepare arbitrary small angle rotation gates with low overheads and small logical error rates. The typical architecture for QEC systems is to have the primary ``computational'' code which maintains all data qubits and external ``factory'' regions which produce special resource states of a single variety, usually T for surface codes. It's been acknowledged that T count and depth dominate resource costs both space and time \cite{litinski2019magic} and alternative proposals have appeared including code switching \cite{anderson2014fault}, higher dimensional codes \cite{kubica2018abcs} and most recently these ``small-angle'' synthesis procedures \cite{akahoshi2023partially, ding2018magic} which propose repeat-until-success (RUS) procedures for the production of \textit{arbitrary} magic states $\ket{m_\theta}$ which can be injected to perform $Rz(\theta)$ directly without the need for expensive decompositions or distillation with varying success. While perhaps not guaranteed to be the way forward, the total space-time cost of these procedures is appealing for near and intermediate term demonstrations of error correction by severely cutting down on total physical qubit requirements that would be necessary for distillation factories. These works propose simple versions of architectures which support their RUS strategies, but they are limited in scale and lack a full compiler to optimize for the non-deterministic behavior of their techniques, which even in other literature has gone largely ignored.

\begin{figure}
    \centering
    \includesvg[width=0.67\columnwidth]{figs/parallel_prepare.svg}
    \caption{Parallel preparation of the $\ket{m_\theta}$ state within a surface code patch of distance $5$}
    \label{fig:parallel_prepare}
\end{figure}
In this work, we focus primarily on the technique and architecture proposed in \cite{akahoshi2023partially} which uses a [[4, 1, 1, 2]] error \textit{detection} code to produce the $\ket{m_\theta}$ which can be embedded into the larger surface code architecture multiple times as in Figure \ref{fig:parallel_prepare}. They give three examples of simple architectures which localize the production of these states: 1. STAR, a 2x2 grid of surface code tiles 1 of which is data and 1 of which produces the resource state and 2 ancilla used for communication, 2. Compact STAR, a 3x1 grid with 1 data and 2 ancilla and 3. Compressed STAR, a 2x1 grid with 1 data and 1 ancilla. Their atomic abstraction is wholly unnecessary and a more complete compiler (this work) can better manage ancilla to both create local and remote magic states and allocate ancilla for communication. We can blend and overlap these tiles in our larger fabric. Each of these can be found in Figure \ref{fig:star}.





% \subsection{Architectural Design}
% % Litinski
% Surface codes with lattice surgery have been extensively studied \cite{litinski2019game} and t

% \section{Motivation: Limits of Traditional Surface Code Compilers}
% % Problems with distllation, resource management
% % Dynamic vs. Static Compilation
% % non-deterministic generation
% % Space overheads in small angle architectures
\section{Realtime Compilation for Continuous Angle Rotation Architectures}
In this section, we discuss the compilation of high level programs to surface code architectures which support Clifford+Rz gates as opposed to the traditional Clifford+T. We assume all programs have already been synthesized into the appropriate gate set. In typical distillation architectures, factories are bulky and hard to intersperse amongst the data qubits. Therefore, resource states are prepared and stored remotely on the boundaries, only to be routed in for consumption as necessary. In contrast, in small angle rotation architectures, only small seeds of $\ket{m_{\theta}}$ states are prepared and expanded locally into surface code qubits. It takes at most a single logical patch for preparation at the cost of additional uncertainty in its preparation time. In prior work, atomic units of data qubits and ancilla for the preparation of single qubit gates are prepared. We, however, propose a much more flexible architecture which allows ancilla to be reused and allocated for various rotations dynamically.

\subsection{Execution of \texorpdfstring{$Rz(\theta)$}{Rz(theta)} Rotation Gate}
As discussed in Section~\ref{section:relatedwork}, the $Rz(\theta)$ rotation gate is executed in two steps, a. preparation of an ancilla qubit in the state $\ket{m_\theta}$, and b. injection of the $\ket{m_\theta}$ state into the data qubit and measurement. If the measurement output signifies a failure (with probability $1/2$), a correction $Rz(2\theta)$ would be required. If this correction fails, another correction gate $Rz(4\theta)$ would be required and so on, in a Repeat-Until-Success (RUS) manner. Specifically, failed injection means a $Rz(-\theta)$ was prepared so a $Rz(2\theta)$ correction would yield the proper rotation. However, since $Rz(2\theta)$ is likely a non-Clifford we must repeat. In general, if an $Rz(2^k\theta)$ injection fails we require $Rz(2^{k+1}\theta)$ correction. Every injection fails with probability $1/2$, hence
\begin{align*}
\mathbb{E}[\text{Num. Injections}] &= \sum_{k=1}^{\infty} k \cdot \text{Pr[k-1 Failures, 1 Success]} \\
&= \sum_{k=1}^{\infty} \frac{k}{2^k} = 2
\end{align*}

In the case where $Rz(2^k\theta)$ is a Clifford for some $k$ (for example consider T or $\sqrt{T}$), this expectation will be $< 2$ since it will no longer require an injection step. There are two potential strategies for injection, summarized in Table~\ref{table:LvsZZ}, one using a CNOT (Figure~\ref{fig:rzrotation}) and one using a ZZ pauli-product measurement (Figure~\ref{fig:pauliprod} with $P = Z$). In either case, the compiler must allocate a contiguous block of ancilla. Our compiler reserves ancilla and generates a schedule for each ancilla (either dynamically or statically) determining when its free; we simply require that some path between $\ket{m_\theta}$ and target all be free at some time.

\begin{figure}
    \centering
    \begin{quantikz}[transparent]
                                            & \gate[wires=5]{M_{P\otimes Z}}  & \gate[wires=4]{P} & \qw\\
                                            &                                 &                   & \qw\\
                                            &                                 &                   & \qw\\
                                            &                                 &                   & \qw\\
    \lstick{$\overline{\ket{m_\theta}}$}    &                                 & \meterD{M_X}\vcw{-1}
\end{quantikz}
    \caption{A Pauli-product measurement circuit}
    \label{fig:pauliprod}
\end{figure}
\begin{figure}
    \centering
    \begin{quantikz}[transparent]
    \lstick{$\overline{\ket{\psi}}$}        & \targ{} & \gate{X} & \rstick{$Rz(\pm\theta)\overline{\ket{\psi}}$}\\
    \lstick{$\overline{\ket{m_\theta}}$}    & \ctrl{-1} & \meterD{M_Z}\vcw{-1}
\end{quantikz}
    \caption{A Rotation gate injection circuit}
    \label{fig:rzrotation}
\end{figure}
\begin{table}
    \centering
    \begin{tabular}{|c|c|c|}
        \hline
        Parameter & CNOT & ZZ \\
        \hline
        Exposed edge & X & Z \\
        Number of ancillas required & 2 & 1 \\
        Lattice surgery cycles needed for injection & 2 & 1 \\ % TODO: @Jonathan please confirm this again
        \hline
    \end{tabular}
    \caption{Difference between the two injection strategies}
    \label{table:LvsZZ}
\end{table}
The preparation of the rotation states $\ket{m_\theta}$ is itself non-deterministic and is prepared in an ancilla patch. Consequently, assigned ancilla are claimed for an indeterminate number of cycles until the state is prepared correctly. Different states $\ket{m_\alpha}$ and $\ket{m_\beta}$ require disjoint ancilla for preparation. Multiple ancilla can be assigned for the preparation of any individual state and any additional successful preparations can be discarded if necessary. The number of ancilla dedicated to the production of a particular state $\ket{m_\theta}$ can dynamically change; for example if $n$ ancilla are assigned in cycle 1 and each fails and some $m$ of these ancilla are needed for other operations, we can reclaim them and in the next cycle try to prepare the state using $n - m > 0$ ancilla. Non-deterministic preparation implies that the exact cycle when consumption can occur is unknown a head of time and motivates eager preparation.

% Thus, any compilation strategy would need to consider which ancillas are good candidates for the preparation (and injection). Additionally, the compiler would need to decide when the preparation starts. Starting the preparation too early would end up hogging resources for another gate operation, since the prepared ancilla is not freed until the injection completes. On the other hand, preparing too late would lead to slower execution of the rotation gate since the data qubit would be waiting on the preparing ancilla. In an ideal scenario, the preparation should begin so that the ancilla is prepared exactly when the data qubit becomes available for the injection.


\subsection{Execution of CNOT Gate}
% routing and rotation
Lattice surgery on surface codes gives us a convenient way to perform CNOT gates between two data qubits that can be arbitrarily far apart in constant time (taking $2$ lattice surgery cycles). The only requirements for the execution is that there should exist a path of ancilla qubits connecting the control to the target (via at least one ancilla qubit), and the path should be from the Z edge of the control qubit to the X edge of the target qubit. It might be possible that such a path does not exist, either due to the ancillas being occupied for another gate operation, or the required edges (X or Z edges) of the data qubit not having any neighboring ancillas (see Section~\ref{section:variableancilla} for an example of this situation). In such cases, the compiler would need to insert an edge-rotation gate to orient the correct edge of the qubit onto the chosen path. This gate operation requires one free ancilla qubit and takes $3$ lattice surgery cycles.\par
The quality and speed of the algorithm used for path computation is thus an essential consideration for the compilation strategy. % The compiler should also be able to insert the edge-rotation gates as and when required, while also minimizing the need for the edge-rotation gates.
For example, Figure~\ref{fig:cnot_selection} shows two different selection algorithms. Both algorithms choose the \textit{shortest} path, however the naive selection of shortest may potentially lead to selecting an ancilla that is being used by a lot of other gate operations and this might lead to delayed start of the CNOT execution since the ancilla might not be free until much later.
\begin{figure}
    \centering
    \begin{subfigure}{0.49\columnwidth}
        \includesvg[width=0.9\linewidth]{figs/cnot_selection_1.svg}
        \caption{Shortest path}
    \end{subfigure}
    \begin{subfigure}{0.49\columnwidth}
        \includesvg[width=0.9\linewidth]{figs/cnot_selection_2.svg}
        \caption{Path with earliest start time}
    \end{subfigure}
    \caption{Different CNOT routing strategies between the two gray data qubits. The darkness of the ancilla qubit indicates the activity of the ancilla.}
    \label{fig:cnot_selection}
\end{figure}

\subsection{Variable Ancilla Availability}\label{section:variableancilla}
% sharing of ancillas across patches
A flexible architecture is considered with a fraction of the data qubits residing in a \textit{compact patch} and the other fraction of the data qubits make up the \textit{regular patch}. Such an architecture also motivates the need to share the ancillas between the qubit patches due to limited resources available to each patch by itself. This allows for better consideration of an actual system with finite resources and non-uniform resource allocation. The compiler needs to take the locally available and shared resources into account when determining which ancillas to use for different operations and how to maximize parallel computation.

\subsection{Static Compilation for Continuous Angle Rotation Architectures}
% the static compilation strategy
\begin{figure}
    \centering
    \scalebox{0.7}{
    \begin{quantikz}[transparent]
    \lstick{$\ket{q_1}$} & \ctrl{6} & \qw & \gate{Rz(\theta_1)} & \qw\\
    \lstick{$\ket{q_3}$} & \qw & \ctrl{2} & \qw & \qw\\
    \lstick{$\ket{q_5}$} & \qw & \qw & \ctrl{3} & \qw\\
    \lstick{$\ket{q_6}$} & \qw & \targ{} & \qw & \qw\\
    \lstick{$\ket{q_7}$} & \qw & \gate{Rz(\theta_7)}\slice{} & \qw & \qw\\
    \lstick{$\ket{q_8}$} & \qw & \qw & \targ{} & \qw\\
    \lstick{$\ket{q_9}$} & \targ{} & \qw & \qw & \qw
\end{quantikz}}
    \caption{Example circuit for execution of the static compiler shown in Figure~\ref{fig:baseline_execution}}
    \label{fig:baseline_circuit}
\end{figure}
\begin{figure*}
    \centering
    \includesvg[width=0.9\textwidth]{figs/baseline_execution.svg}
    \caption{Execution begins at time $t_1$. For layer 1, the ancillas being used for CNOT between $q_1$ and $q_9$ are colored in orange, CNOT between $q_3$ and $q_6$ are colored in yellow, and $Rz$ on $q_7$ are colored in red. Note that the CNOT between $q_3$ and $q_6$ requires to edge-rotate $q_6$ to orient the edges correctly. Completion of the gate execution is indicated by the ancillas turning green. For layer 2, the ancillas being used for the CNOT between $q_5$ and $q_8$ are colored in orange and the $Rz$ on $q_1$ are colored in red.}
    \label{fig:baseline_execution}
\end{figure*}

We consider a static compilation strategy that divides the circuit into layers of gates that can be executed in parallel. The input circuit is divided into layers by iterating on the gates in the order of execution and reserving the ancillas for exactly one gate operation per layer. If a particular gate cannot be executed in the current layer because of lack of ancilla resources, the compiler attempts to execute the gate in the next layer.\par
For the $Rz(\theta)$ gates, we only consider a single ancilla qubit for the preparation, depending on the patch configuration and the exposed edges. For the execution of CNOT gates, the shortest path between the control and target is chosen. While computing the shortest path, only the ancillas that have not been reserved for the current layer are considered. If the need for an edge-rotation is found for the chosen path, edge-rotation gates are inserted before and after the CNOT gate. The rotation gates are inserted after the CNOT gate to ensure that the correct edges continue to remain exposed for the $Rz(\theta)$ gate injection. Choosing the shortest path helps to maximize the parallelism in each layer since the next gate has more ancillas available to choose from. If the control and target cannot be connected, then gate is skipped in the current layer. Figure~\ref{fig:baseline_execution} shows an example execution for the circuit shown in Figure~\ref{fig:baseline_circuit}.

\section{Dynamic Compilation}
In this section, we discuss our dynamic compilation strategy that aims to minimize the total program execution time by executing the gate operations as soon as possible. The performance improvements of this strategy are based on two major observations:
\begin{enumerate}
    \item Multiple ancillas can be prepared in parallel and the preparation can begin before the actual $Rz(\theta)$ gate operation can begin.
    \item The expected `free' time of the ancilla qubits, that is, the time when the ancilla qubits are going to be available for connecting the control and the target, can be used to make instantaneous decisions when determining the best path for the CNOT gate execution
\end{enumerate}
The first observation motivates the need to efficiently manage and allocate the ancillas available to the different gates and data qubits. Since the preparation and injection for the $Rz(\theta)$ rotation gates is done locally, it makes the management feasible with a negligible classical overhead and thus it can be done during runtime without any performance penalty. The exact techniques used are discussed in Section~\ref{section:ancillamanage}.\par
Finding the best path in a graph (the 2D grid of the data qubits and ancillas) such that it can begin execution at the earliest is equivalent to solving the minimax path in a graph. This problem has a polynomial time solution which is obtained by computing the Minimum Spanning Tree and connecting the control and target on this tree. However, the complexity of this algorithm is $O((N + E)\log{E})$ where $N$ is the number of logical qubits in the grid and $E$ is the number of edges. Since we are considering a 2D grid, $E = O(N)$ and therefore the time complexity can simply be written as $O(N\log{N})$. Thus this approach would scale with the size of the grid and would be inefficient. % Indeed, if we consider a classical computation unit that runs at $2$ GHz and assume that we require about $10N\log{N}$ instructions to compute the MST, for a quantum system with $100$ data qubits, it would require between 1-2$\mu s$ to compute the MST for a single CNOT gate. These calculations have been done assuming an IPC (instructions per cycle) of $1$, however, in reality, the IPC of any system is much lower and hence the number of cycles required would increase significantly. Therefore, this leads to a huge overhead and we cannot always choose the best path without introducing huge stalls for each CNOT execution. Instead we propose a strategy that uses the second observation and computes a single tree every $k$ cycles, thus making the computation efficient while also maintaining knowledge of the recent occupancy of the ancilla qubits. We discuss this approach in detail in Section~\ref{section:efficientmst}

\subsection{Intelligent Ancilla Resource Management}\label{section:ancillamanage}
\begin{figure*}
    \centering
    \includesvg[width=0.95\textwidth]{figs/prepare_timeline.svg}
    \caption{An example execution for an $Rz(\theta)$ gate to be executed next on the middle data qubit. The qubit in grey is another data qubit. Time steps $t_1$ and $t_2$ show execution times when different sets of ancilla qubits are busy. $T_i = t_3$ is when it is determined that the data qubit would be free at time $T_d = t_5$ and thus we start preparing all possible ancilla qubits in the neighborhood. Another ancilla qubit becomes free at time $T_a = t_4$ and we start preparing it. At $t_6$, one ancilla qubit succeeds in preparation and thus the injection begins, simultaneously we start preparing ancillas in the state $\ket{m_{2\theta}}$. The injection completes at time $t_7$ and $\ket{\psi'} = Rz(-\theta)\ket{\psi}$, thus we need to perform the correction gate.}
    \label{fig:prepare_timeline}
\end{figure*}

The execution of any gate can be broadly broken down into two steps, preparation and execution. The preparation step only requires the associated ancilla qubits to be free while the execution step requires the data qubit(s), prepared ancilla qubit(s) and any ancilla qubit connecting these qubits to be free. Thus, if we consider the timeline of execution of the gate, the preparation can be done asynchronously. The execution step may or may not require more ancillas apart from the ancilla that was used in the preparation step. Also some gates might not have any preparation step at all.\par
This is especially important for the dynamic compilation of the $Rz(\theta)$ rotation gates. Consider a single ancilla qubit being used for the preparation of the $Rz(\theta)$ gate. For the ancilla and data qubits, we define the parameters $t_\theta$, $T_d$, $T_a$, $T_b$, $T_i$ to understand the dynamic compilation step. $t_\theta$ is the expected time it takes to prepare the ancilla qubit in the state $\ket{m_\theta}$. The time when the data qubit is free and can participate in the execution of the gate operation is denoted by $T_d$. Similarly, the time when the ancilla qubit is free and it can participate in the preparation of the gate operation is denoted by $T_a$. In an ideal scenario, we would want to begin the preparation of the ancilla qubit at time $T_b = \max\left(T_d - t_\theta, T_a\right)$. $T_i$ is the time during the execution of the program when it can be deterministically asserted that the data qubit is going to be free at time $T_d$. % This is important since it might be possible that the total execution time for the $Rz(\theta)$ gate is longer or shorter than the expected time. Let us consider the case when $T_a < T_d - t_\theta$. In this case, we would want to begin the preparation of the ancilla at $T_b = T_d - t_\theta$. However, if $T_i > T_b$, the compiler cannot physically determine that the data qubit can participate in the execution from time $T_d$ and thus it cannot actually begin the preparation until time $T_i$.
Figure~\ref{fig:prepare_timeline} shows an example execution and relevant timestamps.\par
% The more ancilla qubits that are prepared in parallel, the lesser the combined expected time of preparation is. We also permit sharing of the ancilla qubits between different data qubits. Additionally, the paths for connecting distant qubits for CNOT gates also passes through the patches of various data qubits. Therefore we need a dynamic ancilla allocation strategy that can determine when to assign which ancilla to which gate operation. Before we discuss the exact strategy, we note that the ancilla qubits used for the preparation and injection are only those that are present in a $3\times 3$ block with the data qubit in the centre. This is done to simplify the data structures that keep a track of the different ancilla qubits. Additionally, using the ancilla qubits that are far away would lead to making more global decisions and this would lead to increased resource contention since connecting the data qubit to the far away ancilla qubits would require freeing the ancilla qubits on the path.

\subsubsection{Queue for Ancillas}
Another way to reduce the expected execution time of the $Rz(\theta)$ gates is by preparing multiple ancilla qubits in parallel. To make it easy to keep a track of the gate operations that the ancilla qubits will assist, either during the preparation or during the execution, each ancilla qubit maintains a queue of the gate operations that the ancilla has prepare/execute for. The compiler then looks at the head of the queue and begins the required steps necessary for the gate operation. Similarly, when the compiler chooses the relevant ancilla qubits for a particular gate's operations, it pushes the gate to the queue of all the associated ancillas. Each queue entry needs to store some relevant information about the ancilla's role in the preparation and/or execution. In the case of a Rz gate, it might be possible that the ancilla is only being involved for execution, however, it might be associated with two different preparing ancillas. In such scenarios, it might be possible that the ancilla which is not the head of the queue gets prepared first. The compiler then removes all entries in queue that appear before this entry from the queue for this gate even if they appear in the middle of the queue. Most of the queue operations can be performed independently irrespective of the state of the queue of the other ancillas. Therefore, the queues for all ancillas can be maintained in parallel using small computation units for each ancilla with minimal support for coherence.

% \begin{figure}
%     \centering
%     \includesvg[height=0.905\textheight]{figs/queue_timeline_vert.svg}
%     \caption{Execution of the circuit with the queue. The text in bold signifies the current state of the ancilla. The dashed nodes represent the nodes that are being deleted from the queue.}
%     \label{fig:queue_timeline}
% \end{figure}

\subsection{Routing Strategy}\label{section:efficientmst}
\begin{figure}
    \centering
    \includesvg[width=\columnwidth]{figs/mst_selection.svg}
    \caption{Timeline showing the MST computation procedure. The MST computation begins at $t_1$ and the result is available by $t_2$. Another MST computation begins at $t_2$ and its result is available by $t_3$. For simplicity, here $k = p = t_2 - t_1 = t_3 - t_2$}
    \label{fig:mst_selection}
\end{figure}
Now that we have a convenient way to claim ancillas for the gate operations, all that remains is to efficiently determine the path of ancillas that will be used to execute any CNOT gate. To get an estimate of the activity of each ancilla during runtime, the MST algorithm is run on the sub-grid comprising of the ancilla qubits every $k$ cycles, where $k$ is a modifiable parameter. Assuming that the computation of the MST takes about $p$ cycles, the results of the MST computation are \textit{delayed} by $p$ cycles. If we use the weight of each ancilla during the MST computation as the expected time when the ancilla finishes execution, the information would be stale by the time the MST is computed. Thus, instead we use the expected activity of the ancilla in the last $p$ cycles to capture the \textit{heat} (activity) of each ancilla. Now when a CNOT gate requires to be executed, it just looks at the latest available MST and computes up to $4$ different paths. These paths computed are the paths from the $Z\to X$, $X\to Z$, $X\to X$, $Z\to Z$ edges of the control $\to$ target respectively.\par
Note that all paths might not always exist. For the existing paths, the earliest expected start time is computed after accounting for the edge-rotation of the data qubits if the required edges do not align. % Since the edge-rotation requires an ancilla qubit, the ancilla of the chosen path right adjacent to the corresponding data qubit is used for the execution rather than claiming another neighboring ancilla qubit. This is done since the ancilla which is on the chosen path is also on the MST and thus it is the ancilla which is most likely to be freed the earliest.
Once the best path is chosen, the CNOT gate is added to the queue of all the ancilla qubits on the path. Once this CNOT gate reaches the head of all the queues, the execution of the CNOT gate begins. However this approach can lead to increasing load on the same set of ancillas since the same MST is used for multiple CNOT gate executions. This load can be controlled by tuning the parameter $k$. However, reducing the parameter $k$ to a very small number would increase the space required for storing all partially/full computed MSTs. At any point of time, there would be $p/k$ MST computations going on and thus the space required to keep a track of all MSTs would be $O(p/k\cdot N)$ since storing a single tree requires $O(N)$ space. Figure~\ref{fig:mst_selection} shows an example execution of the delayed MST computation.

\section{Evaluation}
\begin{figure*}
    \centering
    \includesvg[width=\textwidth]{figs/plot.svg}
    \caption{Normalized average execution time of different compilers for a variety of benchmarks (the bar in yellow shows the classical stalls)}
    \label{fig:plot}
\end{figure*}

\begin{figure}
    \centering
    \begin{subfigure}{\columnwidth}
        \includegraphics[width=0.9\linewidth]{figs/heatmap_static.png}
        \caption{Static compiler}
    \end{subfigure}
    \begin{subfigure}{\columnwidth}
        \includegraphics[width=0.9\linewidth]{figs/heatmap_dynamic.png}
        \caption{Dynamic compiler ($k = 25$)}
    \end{subfigure}
    \caption{Heatmap of the ancilla qubits for different compilers for the first $1000$ cycles for the QV\_n100 benchmark (the greyed blocks are the data qubits)}
    \label{fig:heatmap}
\end{figure}

We randomly generated a 2D grid with $50\%$ probability of the data qubit being a compact or compressed patch. For the static compiler, the compilation was done before the execution. For the dynamic compiler, compilation was done on-the-fly and the compiler was run for different parameters. We used the QASM Benchmarks from \cite{li2022qasmbench} by choosing circuits with significant number of Rz gates. We transpiled these benchmarks on Qiskit\cite{Qiskit} to the basis gate set of $X, H, Rz, CNOT$ at an optimization level of $2$ and then simulated the execution of the assembled circuits. We tested the circuit with different values of $k$. For the case of $k = 0$, i.e., where we compute the MST for each CNOT operation, we introduced stalls during the execution of the circuit. For all the executions, we assume $p$ (number of cycles it takes to compute the MST) to be $100$. Figure~\ref{fig:plot} contains the plot of execution times for physical qubit error rate $10^{-3}$ and a code distance of $7$.\par
We observe a significant improvement in the execution times for the case of dynamic compiler, even after accounting for the stalls due to MST computation. The improvement is due to maximum utilization of the ancillas during the preparation phase and minimal waiting time for each gate operation. This can also be observed from the heatmap of the ancillas for the total activity in $1000$ cycles as shown in Figure~\ref{fig:heatmap}. Unlike the static case, the ancillas are active for almost half of the epoch duration. It was also observed that the the performance improvements are maintained as the physical error rates are reduced.

\section{Conclusion}
Fault-tolerant quantum architectures are still far from being realised on physical systems. Various error correction techniques have been proposed, surface codes being one of them. The surface code architecture does not natively support non-Clifford gates and thus require specialized support for such operations. Continuous rotation angle architectures have been proposed that allow for localized, low-latency ancilla preparation in the required states. However, this leads to the problem of supporting dynamic ancilla allocation to minimize runtime. Our work tackles this problem and we propose a dynamic compilation technique that utilizes the variable ancilla availability and activity to perform efficient allocation for different gate operations in parallel, thus minimising the total program execution time. We improve significantly over the basseline proposals while simultaneously minimizing classical overhead for realtime recompilation.
% \par
% The ideas proposed in our work are not only limited to compilation for SARA but can directly be used for compilation for any architecture that involves non-deterministic steps and/or variable availability of ancilla resources.


% \section{Introduction}
% TODO: briefly talk about how rotation gates approach is efficient and why non-determinism leads to the requirement for a dynamic scheduler (compiler?).

% \section{Related Work (and Background)}
% TODO: include the related work (STAR architecture, etc.) and the background (quantum computing, etc.)

% \section{Problem and Motivation}
% TODO: expand on the introduction and explain about the issues with static compilation (by showing numbers?)

% \section{Optimal Scheduling/Compilation}
% TODO: discuss the best approach, i.e., the approach with MST (also include small theoretical proofs of correctness?)

% \section{Non-Exact but Faster Design Decisions (TODO: better title)}
% TODO: discuss the trade-offs and why they still work

% \section{Performance Evaluation}
% TODO: describe the experimental setup and show numbers

% \section{Future Work}
% TODO: talk about the future work

% \section{Conclusion}
% TODO: summarize the paper and talk about future work


\bibliographystyle{plain}
\bibliography{references}

\end{document}

