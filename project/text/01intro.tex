\section{Introduction}
\begin{comment}
1. QEC good
2. QEC for Surface Code Appealing for near-term architectures
3. QEC doesn't have a universl gate set inherently
4. Must augment the gate set of the base code with gates from other codes; e.g. T or CCX. THis is done via distillation or code switching
5. Then, use synthesis procedures to convert Rz to S-H-T. This is expensive 
6. Recently new proposals have appeared: generate Rz gates directly \cite{}
7. how does it work?
8. Architectural Support for them is lacking
9. Direct accounting for the non-deterministic behavior of their production and consumption (and possibly long corrections) missing
\end{comment}
\begin{figure*}
    \centering
    \includesvg[width=\textwidth]{figs/intro.svg}
    \caption{Sample execution patterns of STAR \cite{akahoshi2023partially} (top) compared to our more dynamic and efficient resource management compiler (bottom). In STAR, atomic units of 2x2 grids are allocated containing 1 data qubit and 3 ancilla. Continuous angle rotation states $\ket{m_\theta}$ can be prepared locally and consumed within each STAR unit. However, ancilla use is locally restricted as well which means non-deterministic state production can take longer than necessary without utilizing parallel preparations with neighboring ancilla. Our compiler effectively manages ancilla for both state preparation and communication based on dynamically and in realtime computed expected free times for each resource and allow for multiple parallel preparations to minimize total expected execution time, here 6 preparations are attempted in parallel (bottom) as opposed to 1 (top), this essentially guarantees a single cycle to produce the necessary state. This leads to on average $4.5\times$ improvement in total runtime.}
    \label{fig:intro}
\end{figure*}
Current quantum computers very noisy and only able to reliably execute small programs. In order to combat high error rates, quantum error correction is necessary in order to execute large scale applications. Even still, as we scale, quantum resources will be relatively scarce and it is critical to develop real time management of these resources in order to maximize the value of available hardware. These resources include classical bandwidth for decoding \cite{ravi2023better}, ancilla management for communication and decompositions \cite{ding2020square}, and the creation and use of special resource states called \textit{magic states} \cite{ding2018magic, litinski2019magic}. So far, the study of these resources has been concentrated around surface codes as a promising candidate error correction code for small to intermediate scale quantum hardware \cite{litinski2019game, gidney2021factor} due to its limited hardware connectivity, high threshold, and well studied decoding procedures. While surface codes are not definitively the best choice of error correction code, for example qLDPC codes \cite{bravyi2023high, xu2023constant} are a promising choice for high rate codes, they are fairly easy to work since they have well-defined logical operators via lattice surgery \cite{horsman2012surface} and can serve as as testbed for different architectural designs.

Critical to the success of any quantum error correcting code is support for the creation and consumption of so-called \textit{magic states}. Since no code natively supports a transversal and universal gate set \cite{eastin2009restrictions}, there must be some gates that are performed in \textit{another} code and converted or distilled into special states that can be used to apply that gate in the original code. For example, in the surface code we natively only support the Clifford gates (e.g. CNOT, X, Z, and H) which are not universal. One common way to make this gate set universal is to produce T states and inject them as necessary. It is also typical that we only support a finite gate set and not a continuous one. For example, most physical quantum computers can support $Rz(\theta)$ for any $\theta$, where as the surface code only supports a discrete set. However, by adding in the T gate to the Clifford we do get a universal gate set since we can approximate any $Rz(\theta)$ gate to arbitrary precision \cite{ross2014optimal}. Unfortunately, this type of synthesis results in circuits which are extremely long and require large numbers of bulky \textit{factories} to produce the magic states; both space and time requirements for any circuit become dominated by T state production. 

Recently, several alternative approaches have been proposed \cite{akahoshi2023partially, choi2023fault} which rather than requiring magic state factories to create one fixed type of Rz rotation instead propose methods to create analog rotation states $\ket{m_\theta}$, adopting the syntax from \cite{akahoshi2023partially}. When injecting this state we can perform an arbitrary $Rz(\theta)$ rotation. This is especially powerful because it ideally reduces the total space requirement for producing non-Clifford gates since we can use ancilla local to the injection site to prepare the necessary $\ket{m_\theta}$ and also does not require us to distill hundreds or thousands of T gates per $Rz$ rotation. So far, however, there is limited architectural support for this strategy specifically for the realtime management of the nondeterministic behavior of the $\ket{m_\theta}$ production. In the original proposal from \cite{akahoshi2023partially} their STAR architecture provides a basic structure to demonstrate the technique's efficacy, but 1. limits state production to atomic STAR patches which limits parallel production when other space is unused and 2. does not directly adapt program execution as a function of the highly non-deterministic state production. 

In this work, we provide an improved compilation scheme for continuous angle rotation architectures. In our compilation technique, we consider the ancillas independent of the data qubit, thus allowing for sharing of resources between multiple qubits and gates. This warrants the need to make decisions on the dynamic assignment of ancilla since allocating more ancilla for a particular gate operation reduce s the expected time to prepare $\ket{m_\theta}$ but constraints the execution of neighboring gates, whereas allocating fewer ancilla leads to inefficient utilisation of the resources. To counter this problem, we provide a mechanism to dynamically manage the ancilla that are available for different gate operations, including but not limited to Rz and CNOT gates. This approach allows us to flexibly allocate the ancilla for different gates. We also propose an space-time efficient technique that schedules the long-distance gate operations while minimizing the wait time. We achieve an average improvement of $4.5\times$ over a statically compiled execution, even after accounting for classical overheads during dynamic compilation to recompute the best ancilla use. Our approach can be directly incorporated in any quantum architecture involving non-deterministic execution and/or variable ancilla availability.

The primary contributions of this work are
\begin{enumerate}
    \item An efficient compilation scheme for quantum error correction systems which support native continuous angle resource states. We improve over baseline proposals by an average of $4.5\times$ in total program execution time, both quantum and classical delay costs accounted for.
    \item An improved architecture for local resource state production which reduces total space (ancilla) requirements while simultaneously reducing the total runtime of programs on these systems.
    \item Compilation which directly accounts for the inherent non-deterministic behavior of continuous angle resource state production; we introduce the notion of real-time recompilation depending on the prior success of production and consumption of these states.
    \item Real system measurement latencies restrict the amount and frequency of classical recompilation that can occur without incurring excessive idling on the quantum system; our compilation scheme easily adapts to any hardware platform and dynamically selects the frequency of recompilation the first of its kind to do so. This real-time control consideration is extensible to other QEC systems.
\end{enumerate}